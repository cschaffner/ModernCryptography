\documentclass[a4paper,10pt,landscape,twocolumn]{scrartcl}

%% Settings
\newcommand\problemset{6}
\newcommand\deadline{Thursday, 14 September 2017, 20:00h}
\newcommand\worksession{Friday, 29 September 2017}
\newif\ifcomments
\commentsfalse % hide comments
%\commentstrue % show comments

%% Packages
\usepackage[english]{exercises}
\usepackage{wasysym}
\usepackage{hyperref}
\hypersetup{colorlinks=true, urlcolor = blue, linkcolor = blue}
\usepackage{enumitem}

%% Macros
\usepackage{xspace}

\newcommand{\eps}{\varepsilon}
\newcommand{\ket}[1]{|#1\rangle}
\newcommand{\bra}[1]{\langle#1|}
\newcommand{\inp}[2]{\langle{#1}|{#2}\rangle}
\newcommand{\norm}[1]{\parallel\!#1\!\parallel}
\newcommand{\points}[1]{\marginpar{\textbb{#1 p.}}}
\newtheorem{theorem}{Theorem}
\newtheorem{definition}{Definition}
\newtheorem{proposition}{Proposition}
%\newenvironment{proof}{\noindent {\bf Proof }}{{\hfill $\Box$}\\}

\newcommand{\gen}{\ensuremath{\mathsf{Gen}}\xspace}
\newcommand{\enc}{\ensuremath{\mathsf{Enc}}\xspace}
\newcommand{\dec}{\ensuremath{\mathsf{Dec}}\xspace}
\newcommand{\mac}{\ensuremath{\mathsf{Mac}}\xspace}
\newcommand{\vrfy}{\ensuremath{\mathsf{Vrfy}}\xspace}
\newcommand{\negl}{\ensuremath{\mathsf{negl}}\xspace}
\newcommand{\PrivK}{\ensuremath{\mathsf{PrivK}}\xspace}
\newcommand{\eav}{\ensuremath{\mathsf{eav}}\xspace}

\newcommand{\Z}{\ensuremath{\mathbb{Z}}}
\newcommand{\R}{\ensuremath{\mathbb{R}}}
\newcommand{\N}{\ensuremath{\mathbb{N}}}


\newcommand\floor[1]{\lfloor#1\rfloor}
\newcommand\ceil[1]{\lceil#1\rceil}

% \newcommand{\comment}[1]{{\sf [#1]}\marginpar[\hfill !!!]{!!!}}
\newcommand{\chris}[1]{\comment{\color{blue}Chris: #1}}
\newcommand{\jan}[1]{\comment{\color{magenta}Jan: #1}}


\begin{document}

\problems

{\sffamily\noindent
We will work on the following exercises together during the work sessions on \worksession.

You are strongly encouraged to work together on the exercises, including the homework. You do not have to hand in solutions to these problem sets.}

\begin{exercise}[Exercise~3.20 from  $\text{[KL]}$ ]
  Consider a stateful variant of CBC-mode encryption where the sender simply increments the $IV$ by 1 each time a message is encrypted (rather than choosing $IV$ at random each time). Show that the resulting scheme is \emph{not} CPA-secure.
\end{exercise}

\begin{exercise}[Exercise~3.28 from $\text{[KL]}$ ]
Show that the CBC, OFB, and CTR modes of operation do not yield CCA-secure encryption schemes (regardless of $F$).
\end{exercise}

\begin{exercise}[Exercise~3.29 from $\text{[KL]}$ ]
  Let $\Pi_1 = (\enc_1, \dec_1)$ and $\Pi_2 = (\enc_2, \dec_2)$ be two encryption schemes for which it is known that at least one is CPA-secure (but you don’t know which one). Show how to construct an encryption scheme $\Pi$ that is guaranteed to be CPA-secure as long as at least one of $\Pi_1$ or $\Pi_2$ is CPA-secure. Provide a full proof of your solution.

  \textbf{Hint:} Generate two plaintext messages from the original plaintext so that knowledge of either one reveals nothing about the original plaintext, but knowledge of both enables the original plaintext to be computed.
\end{exercise}

\end{document}
