\documentclass[a4paper,10pt,landscape,twocolumn]{scrartcl}

%% Settings
\newcommand\problemset{7}
\newcommand\worksession{Tuesday, 16 October 2018}
\newif\ifcomments
\commentsfalse % hide comments
%\commentstrue % show comments

%% Packages
\usepackage[english]{exercises}
\usepackage{wasysym}
\usepackage{hyperref}
\hypersetup{colorlinks=true, urlcolor = blue, linkcolor = blue}
\usepackage{enumitem}
\usepackage{graphicx}

%% Macros
\usepackage{xspace}

\newcommand{\eps}{\varepsilon}
\newcommand{\ket}[1]{|#1\rangle}
\newcommand{\bra}[1]{\langle#1|}
\newcommand{\inp}[2]{\langle{#1}|{#2}\rangle}
\newcommand{\norm}[1]{\parallel\!#1\!\parallel}
\newcommand{\points}[1]{\marginpar{\textbb{#1 p.}}}
\newtheorem{theorem}{Theorem}
\newtheorem{definition}{Definition}
\newtheorem{proposition}{Proposition}
%\newenvironment{proof}{\noindent {\bf Proof }}{{\hfill $\Box$}\\}

\newcommand{\gen}{\ensuremath{\mathsf{Gen}}\xspace}
\newcommand{\enc}{\ensuremath{\mathsf{Enc}}\xspace}
\newcommand{\dec}{\ensuremath{\mathsf{Dec}}\xspace}
\newcommand{\mac}{\ensuremath{\mathsf{Mac}}\xspace}
\newcommand{\vrfy}{\ensuremath{\mathsf{Vrfy}}\xspace}
\newcommand{\negl}{\ensuremath{\mathsf{negl}}\xspace}
\newcommand{\PrivK}{\ensuremath{\mathsf{PrivK}}\xspace}
\newcommand{\eav}{\ensuremath{\mathsf{eav}}\xspace}

\newcommand{\Z}{\ensuremath{\mathbb{Z}}}
\newcommand{\R}{\ensuremath{\mathbb{R}}}
\newcommand{\N}{\ensuremath{\mathbb{N}}}


\newcommand\floor[1]{\lfloor#1\rfloor}
\newcommand\ceil[1]{\lceil#1\rceil}

% \newcommand{\comment}[1]{{\sf [#1]}\marginpar[\hfill !!!]{!!!}}
\newcommand{\chris}[1]{\comment{\color{blue}Chris: #1}}
\newcommand{\jan}[1]{\comment{\color{magenta}Jan: #1}}




\begin{document}

\problems

\begin{exercise}[RSA Signatures]
\begin{subex}
Consider the plain RSA signature scheme with modulus $N = 85 = 5 \cdot 17$.
\begin{enumerate}
\item  Compute $\phi(N)$.
\item  Say the public exponent is $e = 5$. Find the private exponent $d$. (Hint: look at
the sequence $\phi(N)+1, 2\cdot \phi(N)+1, \ldots$ until you find a multiple of $e$.)
\item  Compute the signature on the message $3$.\\\textbf{Hint: } Note that $[3^4 \mod 85] = [81 \mod 85] = [(-4) \mod 85]$.
\end{enumerate}
\end{subex}
\begin{subex}
Consider a \emph{padded} RSA signature scheme, where to sign a message $m \in \{0, 1\}^{80}$
the signer chooses random $r$ with $r \| m < N$ (where $\|$ denotes concatenation),
computes $\sigma = [(r \| m)^d \mod N ]$, and outputs signature $\sigma$.
\begin{enumerate}
\item  How would verification be done?
\item  Is this scheme secure? If yes, give a 1-2 sentence explanation; if not, show
an attack.
\end{enumerate}
\end{subex}
\end{exercise}


\begin{exercise}[Not a PRF]
Consider the keyed function $H : \{0, 1\}^n \times \{0, 1\}^n \rightarrow \{0, 1\}^{2n}$ defined as: $H_k(x) = G(k) \oplus G(x)$, where $G: \{0, 1\}^n \rightarrow \{0, 1\}^{2n}$ is a pseudorandom generator.
\begin{subex}
Describe and formally analyze an explicit attack showing that $H$ is not a PRF.
\end{subex}
\begin{subex}
Is there a successful attack making a single query that distinguishes $H_k$ (for random $k$) from a random function $f : \{0, 1\}^n \rightarrow \{0, 1\}^{2n}$? Why or why not?
\end{subex}
\end{exercise}



\begin{exercise}[A randomized variable-length MAC from a PRF]

Let $F$ be a pseudorandom function. Show that the following MAC is insecure for variable-length messages. $\mathsf{Gen}$ outputs a uniform $k\in\{0,1\}^n$. Let $\langle i\rangle$ denote an $n/2$-bit encoding of the integer $i$.

To authenticate a message $m=m_1\|\dots\| m_{\ell}$, where $m_i\in\{0,1\}^{n/2}$, choose a uniform $r\gets\{0,1\}^n$, compute $t:=F_k(r)\oplus F_k(\langle 1\rangle\Vert m_1)\oplus\cdots\oplus F_k(\langle \ell\rangle\Vert m_{\ell})$ and let the tag be $( r,t)$.

\end{exercise}


\begin{exercise}[Cryptographic Mechanisms]
For each of the following, identify the most appropriate cryptographic
mechanism(s) (from among private-key encryption, pseudorandom generators, pseudorandom functions, message authentication codes, hash functions, public-key encryption, or digital signatures) for addressing the problem. Points will be deducted if you
list extraneous mechanisms. \textbf{Explain your answer in 1-2 sentences.}
\begin{subex}
A company wants to distribute authenticated software updates to its customers.
\end{subex}
\begin{subex}
A user wants to ensure secrecy of the files stored on his hard drive.
\end{subex}
\begin{subex}
A customer wants to send his credit card number (confidentially) to a merchant
over the web to complete a purchase.
\end{subex}
\begin{subex}
A general wants to send a message to a lieutenant, and wants to ensure both
confidentiality and integrity.
\end{subex}
\begin{subex}
A client wants to store a short record of a large file he uploads to a server, so that
the client can verify that the file has not been altered when it downloads the file
later.
\end{subex}
\begin{subex}
A user needs 1,000,000 random bits in order to run a simulation, but obtaining
truly random bits is expensive.
\end{subex}
\end{exercise}

\begin{exercise}[Mode of Encryption]
Let $F : \{0, 1\}^n \times \{0, 1\}^n \rightarrow \{0, 1\}^n$ be a block cipher, and consider the
following mode of encryption: to encrypt an $\ell$-block message $m_1, \ldots , m_{\ell}$ using key $k$,
choose uniform $c_0 \in \{0, 1\}^n$ and then for $i = 1,  \ldots , \ell$ set $c_i := F_k(m_i) \oplus c_{i-1}$. Output
the ciphertext $c_0 , \ldots c_{\ell}$.
\begin{subex}
How would decryption of a ciphertext $c_0,  \ldots , c_{\ell}$ be done?
\end{subex}
\begin{subex}
Describe and analyze an explicit attack showing that this scheme is not EAV-
secure.
\end{subex}
\begin{subex}
Is this scheme CPA-secure? Provide a brief justification of your answer.
\end{subex}
\end{exercise}

\begin{exercise}[Padded RSA]
	Let $\tilde \Pi=(\tilde\gen, \tilde \enc, \tilde \dec)$ be the plain RSA encryption scheme for $2 n$ bit messages, and consider the
	padded encryption scheme $\Pi=(\gen, \enc, \dec)$ where $\gen=\tilde\gen$. To encrypt a plaintext $m\in\{0,1\}^n$, sample $r\gets\{0,1\}^n$ and output $\tilde \enc_{\mathsf{pk}}(x\|r)$. Decryption is done by decrypting with $\tilde\dec_{\mathsf{sk}}$ and outputting the first half of the resulting string.
	\begin{subex}
		Find a chosen ciphertext attack on $\Pi$. Give a precise description of an adversary $\mathcal A$, using the notation introduced for the indistinguishability experiments. Avoid imprecise verbose descriptions. Calculate the success probability $\mathcal A$.
	\end{subex}
\end{exercise}


\end{document}
