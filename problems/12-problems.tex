\documentclass[a4paper,10pt,landscape,twocolumn]{scrartcl}

%% Settings
\newcommand\problemset{12}
\newcommand\worksession{Friday, 20 October 2017}
\newif\ifcomments
\commentsfalse % hide comments
%\commentstrue % show comments

%% Packages
\usepackage[english]{exercises}
\usepackage{wasysym}
\usepackage{hyperref}
\hypersetup{colorlinks=true, urlcolor = blue, linkcolor = blue}
\usepackage{enumitem}

%% Macros
\usepackage{xspace}

\newcommand{\eps}{\varepsilon}
\newcommand{\ket}[1]{|#1\rangle}
\newcommand{\bra}[1]{\langle#1|}
\newcommand{\inp}[2]{\langle{#1}|{#2}\rangle}
\newcommand{\norm}[1]{\parallel\!#1\!\parallel}
\newcommand{\points}[1]{\marginpar{\textbb{#1 p.}}}
\newtheorem{theorem}{Theorem}
\newtheorem{definition}{Definition}
\newtheorem{proposition}{Proposition}
%\newenvironment{proof}{\noindent {\bf Proof }}{{\hfill $\Box$}\\}

\newcommand{\gen}{\ensuremath{\mathsf{Gen}}\xspace}
\newcommand{\enc}{\ensuremath{\mathsf{Enc}}\xspace}
\newcommand{\dec}{\ensuremath{\mathsf{Dec}}\xspace}
\newcommand{\mac}{\ensuremath{\mathsf{Mac}}\xspace}
\newcommand{\vrfy}{\ensuremath{\mathsf{Vrfy}}\xspace}
\newcommand{\negl}{\ensuremath{\mathsf{negl}}\xspace}
\newcommand{\PrivK}{\ensuremath{\mathsf{PrivK}}\xspace}
\newcommand{\eav}{\ensuremath{\mathsf{eav}}\xspace}

\newcommand{\Z}{\ensuremath{\mathbb{Z}}}
\newcommand{\R}{\ensuremath{\mathbb{R}}}
\newcommand{\N}{\ensuremath{\mathbb{N}}}


\newcommand\floor[1]{\lfloor#1\rfloor}
\newcommand\ceil[1]{\lceil#1\rceil}

% \newcommand{\comment}[1]{{\sf [#1]}\marginpar[\hfill !!!]{!!!}}
\newcommand{\chris}[1]{\comment{\color{blue}Chris: #1}}
\newcommand{\jan}[1]{\comment{\color{magenta}Jan: #1}}




\begin{document}

\problems

{\sffamily\noindent
We will work on the following exercises together during the work sessions on \worksession.

You are strongly encouraged to work together on the exercises, including the homework. You do not have to hand in solutions to these problem sets.}

\begin{exercise}[Plain RSA Signatures]
Say the public key is $\langle N,e \rangle=\langle 91,11\rangle$.
\begin{enumerate}
\item Compute $\phi(N)$.
\item Calculate the private key $d$.
\item Sign the message $m=28$.
\item Verify that $(43,36)$ is a valid message-signature pair.
\end{enumerate}

\end{exercise}


\begin{exercise}[Insecurity of plain RSA Signatures]

In Section 12.4.1 we showed an attack on the
plain RSA signature scheme in which an attacker forges a signature
on an arbitrary message using two signing queries. Show how an
attacker can forge a signature on an arbitrary message using a
\emph{single} signing query.

\textbf{Hint:} Use the no-query and the two-query attacks. 
\end{exercise}


\begin{exercise}[Plain RSA Signatures, weaker definition]
Assume the RSA problem is hard. Show that the plain RSA signature scheme satisfies the following weak definition of security: an attacker is given the public key $\langle N, e\rangle$ and a uniform message $m \in \mathbb{Z}^*_N$. The adversary succeeds if it can output a valid signature on $m$ without making any signing queries.
\end{exercise}



\begin{exercise}[One-time secure signature scheme?]
A signature scheme is one-time-secure if no PPT adversary making a \emph{single} query can output a valid forgery.

Let $f$ be a one-way permutation (it is hard to calculate the inverse of $f$). Consider the following signature
scheme for messages in the set $\{1,\dots , n\}$:
\begin{itemize}
\item To generate keys, choose uniform $x \in \{0, 1\}^n$ and set $y := f^{(n)} (x)$
(where $f^{(i)}(.)$ refers to $i$-fold iteration of $f$, and $f^{(0)} (x) = x$). The
public key is $y$ and the private key is $x$.
\item To sign message $i \in \{1,\dots , n\}$, output $f^{(n-i)} (x)$.
\item To verify signature $\sigma$ on message $i$ with respect to public key $y$,
check whether $y = f^{(i)} (\sigma)$.
\end{itemize}

\begin{enumerate}
\item Show that the verification procedure will output 1 for every legal message-signature pair.
\item Show that the above is not a one-time-secure signature scheme.
Given a signature on a message $i$, for what messages $j$ can an
adversary output a forgery?
\item Prove that no PPT adversary given a signature on $i$ can output a
forgery on any message $j > i$ except with negligible probability.
\end{enumerate}
\end{exercise}


\newpage

\begin{bonusexercise}[Another one-time secure signature scheme]
Let $f$ be a permutation and $f^{(i)}(x)$ the $i$-fold iteration of $f$,
and $f^{(0)}(x) := x$. Let us consider the following signature scheme
$\Pi = (\mathsf{Gen},\mathsf{Sign},\mathsf{Vrfy})$ for messages $m \in \{1, \ldots, p\}$ with $p = p(n)$ polynomial in $n$.
\begin{center}\vspace{-1em}
\begin{tabular}{rcl}
  $\mathsf{Gen}(1^n)$ & : &
    Choose $\mathsf{sk}_1,\mathsf{sk}_2 \in_R \{0,1\}^n$, $\mathsf{pk}_1 := f^p(\mathsf{sk}_1)$ and $\mathsf{pk}_2 := f^p(\mathsf{sk}_2)$. \\
    & & Set $\mathsf{sk} := (\mathsf{sk}_1,\mathsf{sk}_2)$ and $\mathsf{pk} := (\mathsf{pk}_1,\mathsf{pk}_2)$.\\[5pt]
  $\mathsf{Sign}_{\mathsf{sk}}(m)$ & : &
    Compute $\sigma_1 := f^{(p-m)}(\mathsf{sk_1})$ and $\sigma_2 := f^{(m-1)}(\mathsf{sk}_2)$.\\
    && Return $\sigma := (\sigma_1,\sigma_2)$.\\[5pt]
  $\mathsf{Vrfy}_{\mathsf{pk}}(m,\sigma)$ & : &
    If $\mathsf{pk}_1 =
  f^{(m)}(\sigma_1)$ and $\mathsf{pk}_2 = f^{(p-m+1)}(\sigma_2)$ return $1$,
  else return $0$.
\end{tabular}
\end{center}
\begin{enumerate}
\item Show that $\Pi$ is correct.
\item Prove that $\Pi$ is a one-time-secure signature scheme, if
  $f$ is a one-way permutation.
\end{enumerate}
\end{bonusexercise}


\end{document}
