\documentclass[a4paper,10pt,landscape,twocolumn]{scrartcl}

%% Settings
\newcommand\problemset{5}
\newcommand\worksession{Tuesday, 10 October 2017}
\newif\ifcomments
\commentsfalse % hide comments
%\commentstrue % show comments

%% Packages
\usepackage[english]{exercises}
\usepackage{wasysym}
\usepackage{hyperref}
\hypersetup{colorlinks=true, urlcolor = blue, linkcolor = blue}
\usepackage{enumitem}

%% Macros
\usepackage{xspace}

\newcommand{\eps}{\varepsilon}
\newcommand{\ket}[1]{|#1\rangle}
\newcommand{\bra}[1]{\langle#1|}
\newcommand{\inp}[2]{\langle{#1}|{#2}\rangle}
\newcommand{\norm}[1]{\parallel\!#1\!\parallel}
\newcommand{\points}[1]{\marginpar{\textbb{#1 p.}}}
\newtheorem{theorem}{Theorem}
\newtheorem{definition}{Definition}
\newtheorem{proposition}{Proposition}
%\newenvironment{proof}{\noindent {\bf Proof }}{{\hfill $\Box$}\\}

\newcommand{\gen}{\ensuremath{\mathsf{Gen}}\xspace}
\newcommand{\enc}{\ensuremath{\mathsf{Enc}}\xspace}
\newcommand{\dec}{\ensuremath{\mathsf{Dec}}\xspace}
\newcommand{\mac}{\ensuremath{\mathsf{Mac}}\xspace}
\newcommand{\vrfy}{\ensuremath{\mathsf{Vrfy}}\xspace}
\newcommand{\negl}{\ensuremath{\mathsf{negl}}\xspace}
\newcommand{\PrivK}{\ensuremath{\mathsf{PrivK}}\xspace}
\newcommand{\eav}{\ensuremath{\mathsf{eav}}\xspace}

\newcommand{\Z}{\ensuremath{\mathbb{Z}}}
\newcommand{\R}{\ensuremath{\mathbb{R}}}
\newcommand{\N}{\ensuremath{\mathbb{N}}}


\newcommand\floor[1]{\lfloor#1\rfloor}
\newcommand\ceil[1]{\lceil#1\rceil}

% \newcommand{\comment}[1]{{\sf [#1]}\marginpar[\hfill !!!]{!!!}}
\newcommand{\chris}[1]{\comment{\color{blue}Chris: #1}}
\newcommand{\jan}[1]{\comment{\color{magenta}Jan: #1}}




\begin{document}

\problems

\begin{exercise}[Fermat's little theorem and modular arithmetic.]
\begin{subex}[Modular roots]
Show that the 7th root of 47 modulo 143 is $[47^{103} \mod 143]$. Note that $143=11\cdot 13$.
\end{subex}

\begin{subex}[Computing seemingly huge numbers by hand]
Compute the final two (decimal) digits of $3^{1000}$ (by hand).

\textbf{Hint:} The answer is $[3^{1000} \mod 100]$. If $\text{gcd}(a,b)=1$, then $\phi(a\cdot b)=\phi(a)\cdot\phi(b)$.
\end{subex}

\begin{subex}[Modular arithmetic.]
Let $p$,$N$ be integers with $p|N$ (i.e.\ $p$ divides $N$). Prove that for any integer $X$,
\begin{equation*}
[[X \mod N]\mod p]=[X \mod p].
\end{equation*}
Show that, in contrast, $[[X \mod p]\mod N]$ need not equal $[X\mod N]$.
\end{subex}

\end{exercise}

%\begin{exercise}[Modular roots]
%Show that the 7th root of 47 modulo 143 is $[47^{103} \mod 143]$. Note that $143=11\cdot 13$.
%\end{exercise}

%\begin{exercise}[Computing seemingly huge numbers by hand]
%Compute the final two (decimal) digits of $3^{1000}$ (by hand).
%
%\textbf{Hint:} The answer is $[3^{1000} \mod 100]$. If $\text{gcd}(a,b)=1$, then $\phi(a\cdot b)=\phi(a)\cdot\phi(b)$.
%\end{exercise}

%\begin{exercise}[Generators of the multiplicative group $\mathbb{Z}^*_{11}$.]
%The multiplicative group $\mathbb{Z}^*_{11}$ is a cyclic group. Give all possible generators of this group.
%\end{exercise}

\begin{exercise}[Discrete logarithms]
\begin{subex}[Easy discrete-logarithm problem.]
Explain why the discrete-logarithm problem in the additive group $(\mathbb{Z}_N, +)$ generated by $\langle 1\rangle$ is easy to solve.
\end{subex}

\begin{subex}[Diffie-Hellman problem]
Let us consider the cyclic group $\mathbb{Z}_{13}^*$.
\begin{enumerate}
\item Show that 2 is a generator of $\mathbb{Z}_{13}^*$.
\item In $\mathbb{Z}_{13}^*$, it holds that 
\[
\mathsf{DH}_2(6,9) = \mathsf{DH}_2(2^5, 2^8) = 2^{40} = 2^{40 \mod 12} = 2^4 = 3
\]
Show in the same way that $\mathsf{DH}_2(12,10)=1$.
\end{enumerate}

\end{subex}

\end{exercise}

%\begin{exercise}[Easy discrete-logarithm problem.]
%Explain why the discrete-logarithm problem in the additive group $(\mathbb{Z}_N, +)$ generated by $\langle 1\rangle$ is easy to solve.
%\end{exercise}

%\begin{exercise}[Diffie-Hellman problem]
%Let us consider the cyclic group $\mathbb{Z}_{13}^*$.
%\begin{subex}
%Show that 2 is a generator of $\mathbb{Z}_{13}^*$.
%\end{subex}
%
%\begin{subex}
%In $\mathbb{Z}_{13}^*$, it holds that 
%\[
%\mathsf{DH}_2(6,9) = \mathsf{DH}_2(2^5, 2^8) = 2^{40} = 2^{40 \mod 12} = 2^4 = 3
%\]
%Show in the same way that $\mathsf{DH}_2(12,10)=1$.
%\end{subex}
%\end{exercise}


%\begin{exercise}[Modular arithmetic.]
%Let $p$,$N$ be integers with $p|N$ (i.e.\ $p$ divides $N$). Prove that for any integer $X$,
%\begin{equation*}
%[[X \mod N]\mod p]=[X \mod p].
%\end{equation*}
%Show that, in contrast, $[[X \mod p]\mod N]$ need not equal $[X\mod N]$.
%\end{exercise}

%\begin{exercise}[Computational Diffie-Hellman]
%Define the hardness of the Computational Diffie-Hellman problem with respect to the group-generation algorithm $\mathcal{G}$. %Show your answer to one of the teachers.
%\end{exercise}

\begin{exercise}[Diffie-Hellman]
\begin{subex}[Computational Diffie-Hellman]
Define the hardness of the Computational Diffie-Hellman problem with respect to the group-generation algorithm $\mathcal{G}$. %Show your answer to one of the teachers.
\end{subex}
\begin{subex}[DDH is hard $\Rightarrow$ CDH is hard $\Rightarrow$ DLog is hard]
Prove that hardness of the CDH problem implies hardness of the discrete-logarithm problem, and that hardness of the Decisional Diffie-Hellman problem implies hardness of the CDH problem.
\end{subex}
\end{exercise}




\begin{exercise}[RSA]
\begin{subex}
Let $N=pq$ be a RSA-modulus and let
  $(N,e,d) \leftarrow \mathsf{GenRSA}$. 
Show that the ability of efficiently factoring $N$ allows to compute $d$
  efficiently.
  
  Given 
  \begin{equation*}
  N=2140310672120493293362298457402658192652108411554313695782475927669427
  \end{equation*}
  try to compute $\phi(N)$. If that takes too long compute $\phi(p\cdot q)$, where 
  \begin{align*}
  p=&58240080352490526776497122885950201 \, ,\\
  q=&36749789134330529121473391864214027 \, .
  \end{align*}
  Knowing that compute $d$ for given $N$ and $e=11$. Is it 
  \begin{equation*}
  1945736974654993903056634961275143725147489931575688907101782888641091 \  ?
  \end{equation*}
\end{subex}

\begin{subex}[Factoring RSA Moduli]
Let $N=pq$ be a RSA-modulus and let
 $(N,e,d) \leftarrow \mathsf{GenRSA}$. In this exercise, you show
 that for the special case of $e=3$, computing $d$ is equivalent to
 factoring $N$. Show the following:
 \begin{enumerate}
 \item The ability of efficiently factoring $N$ allows to compute $d$
 efficiently. This shows one implication.
 \item Given $\phi(N)$ and $N$, show how to compute $p$ and $q$. 
\textbf{Hint:} Derive a quadratic equation (over the integers) in the
   unknown $p$.
   \item Assume we know $e=3$ and $d \in \{1,2,\ldots,\phi(N)-1\}$ such
   that $ed \equiv 1 \mod \phi(N)$. Show how to efficiently compute
   $p$ and $q$. \textbf{Hint:} Obtain a small list of possibilities for
   $\phi(N)$ and use 2.
   \item Given $e=3$, $d=29'531$ and $N=44'719$, factor $N$ using the
 method above.
 \end{enumerate}
\end{subex}
\end{exercise}
%\begin{exercise}[RSA]
%Let $N=pq$ be a RSA-modulus and let
%  $(N,e,d) \leftarrow \mathsf{GenRSA}$. 
%Show that the ability of efficiently factoring $N$ allows to compute $d$
%  efficiently.
%  
%  Given 
%  \begin{equation*}
%  N=2140310672120493293362298457402658192652108411554313695782475927669427
%  \end{equation*}
%  try to compute $\phi(N)$. If that takes too long compute $\phi(p\cdot q)$, where 
%  \begin{align*}
%  p=&58240080352490526776497122885950201 \, ,\\
%  q=&36749789134330529121473391864214027 \, .
%  \end{align*}
%  Knowing that compute $d$ for given $N$ and $e=11$. Is it 
%  \begin{equation*}
%  1945736974654993903056634961275143725147489931575688907101782888641091 \  ?
%  \end{equation*}
%\end{exercise}

%\begin{bonusexercise}[DDH is hard $\Rightarrow$ CDH is hard $\Rightarrow$ DLog is hard]
%Prove that hardness of the CDH problem implies hardness of the discrete-logarithm problem, and that hardness of the Decisional Diffie-Hellman problem implies hardness of the CDH problem.
%\end{bonusexercise}

%\begin{bonusexercise}[Factoring RSA Moduli]
%Let $N=pq$ be a RSA-modulus and let
% $(N,e,d) \leftarrow \mathsf{GenRSA}$. In this exercise, you show
% that for the special case of $e=3$, computing $d$ is equivalent to
% factoring $N$. Show the following:
%\begin{subex}
%The ability of efficiently factoring $N$ allows to compute $d$
% efficiently. This shows one implication.
%\end{subex}
%
%\begin{subex}
%Given $\phi(N)$ and $N$, show how to compute $p$ and $q$. 
%\textbf{Hint:} Derive a quadratic equation (over the integers) in the
%   unknown $p$.
%\end{subex}
% 
%\begin{subex}
%Assume we know $e=3$ and $d \in \{1,2,\ldots,\phi(N)-1\}$ such
%   that $ed \equiv 1 \mod \phi(N)$. Show how to efficiently compute
%   $p$ and $q$. \textbf{Hint:} Obtain a small list of possibilities for
%   $\phi(N)$ and use (b).
%\end{subex}
%
%\begin{subex} 
%Given $e=3$, $d=29'531$ and $N=44'719$, factor $N$ using the
% method above.
%\end{subex}
%\end{bonusexercise}


\begin{exercise}[Man-In-The-Middle Attacks]
Describe in detail a man-in-the-middle attack on the Diffie-Hellman key-exchange protocol whereby the adversary ends up sharing a key $k_A$ with Alice and a (different) key $k_B$ with Bob, and Alice and Bob cannot detect that anything has gone wrong.
What happens if Alice and Bob try to detect the presence of a man-in- the-middle adversary by sending each other (encrypted) questions that only the other party would know how to answer?
\end{exercise}

\begin{exercise}[Key Exchange with Bit Strings]
  Consider the following key-exchange protocol:
\begin{enumerate}
\item Alice chooses $k,r \leftarrow \{0,1\}^n$ at random, and sends $s := k \oplus r$ to Bob.
\item Bob chooses $t\leftarrow \{0,1\}^n$ at random and sends
  $u:=s \oplus t$ to Alice.
\item Alice computes $w := u \oplus r$ and sends $w$ to Bob.
\item Alice outputs $k$ and Bob computes $w \oplus t$.
\end{enumerate}
Show that Alice and Bob output the same key. Analyze the security of
the scheme (i.e., either prove its security or show a concrete
attack).
\end{exercise}



\end{document}
