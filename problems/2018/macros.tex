\usepackage{xspace}

\newcommand{\eps}{\varepsilon}
\newcommand{\ket}[1]{|#1\rangle}
\newcommand{\bra}[1]{\langle#1|}
\newcommand{\inp}[2]{\langle{#1}|{#2}\rangle}
\newcommand{\norm}[1]{\parallel\!#1\!\parallel}
\newcommand{\points}[1]{\marginpar{\textbb{#1 p.}}}
\newtheorem{theorem}{Theorem}
\newtheorem{definition}{Definition}
\newtheorem{proposition}{Proposition}
%\newenvironment{proof}{\noindent {\bf Proof }}{{\hfill $\Box$}\\}

\newcommand{\gen}{\ensuremath{\mathsf{Gen}}\xspace}
\newcommand{\enc}{\ensuremath{\mathsf{Enc}}\xspace}
\newcommand{\dec}{\ensuremath{\mathsf{Dec}}\xspace}
\newcommand{\mac}{\ensuremath{\mathsf{Mac}}\xspace}
\newcommand{\vrfy}{\ensuremath{\mathsf{Vrfy}}\xspace}
\newcommand{\negl}{\ensuremath{\mathsf{negl}}\xspace}
\newcommand{\PrivK}{\ensuremath{\mathsf{PrivK}}\xspace}
\newcommand{\eav}{\ensuremath{\mathsf{eav}}\xspace}

\newcommand{\A}{\ensuremath{\mathcal{A}}}

\newcommand{\Z}{\ensuremath{\mathbb{Z}}}
\newcommand{\R}{\ensuremath{\mathbb{R}}}
\newcommand{\N}{\ensuremath{\mathbb{N}}}


\newcommand\floor[1]{\lfloor#1\rfloor}
\newcommand\ceil[1]{\lceil#1\rceil}

% \newcommand{\comment}[1]{{\sf [#1]}\marginpar[\hfill !!!]{!!!}}
\newcommand{\chris}[1]{\comment{\color{blue}Chris: #1}}
\newcommand{\jan}[1]{\comment{\color{magenta}Jan: #1}}
