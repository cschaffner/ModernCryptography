\documentclass[a4paper,10pt,landscape,twocolumn]{scrartcl}

%% Settings
\newcommand\problemset{4}
\newcommand\deadline{Thursday, 14 September 2017, 20:00h}
\newif\ifcomments
\commentsfalse % hide comments
%\commentstrue % show comments

%% Packages
\usepackage[english]{exercises}
\usepackage{wasysym}
\usepackage{hyperref}
\hypersetup{colorlinks=true, urlcolor = blue, linkcolor = blue}
\usepackage{enumitem}

%% Macros
\usepackage{xspace}

\newcommand{\eps}{\varepsilon}
\newcommand{\ket}[1]{|#1\rangle}
\newcommand{\bra}[1]{\langle#1|}
\newcommand{\inp}[2]{\langle{#1}|{#2}\rangle}
\newcommand{\norm}[1]{\parallel\!#1\!\parallel}
\newcommand{\points}[1]{\marginpar{\textbb{#1 p.}}}
\newtheorem{theorem}{Theorem}
\newtheorem{definition}{Definition}
\newtheorem{proposition}{Proposition}
%\newenvironment{proof}{\noindent {\bf Proof }}{{\hfill $\Box$}\\}

\newcommand{\gen}{\ensuremath{\mathsf{Gen}}\xspace}
\newcommand{\enc}{\ensuremath{\mathsf{Enc}}\xspace}
\newcommand{\dec}{\ensuremath{\mathsf{Dec}}\xspace}
\newcommand{\mac}{\ensuremath{\mathsf{Mac}}\xspace}
\newcommand{\vrfy}{\ensuremath{\mathsf{Vrfy}}\xspace}
\newcommand{\negl}{\ensuremath{\mathsf{negl}}\xspace}
\newcommand{\PrivK}{\ensuremath{\mathsf{PrivK}}\xspace}
\newcommand{\eav}{\ensuremath{\mathsf{eav}}\xspace}

\newcommand{\Z}{\ensuremath{\mathbb{Z}}}
\newcommand{\R}{\ensuremath{\mathbb{R}}}
\newcommand{\N}{\ensuremath{\mathbb{N}}}


\newcommand\floor[1]{\lfloor#1\rfloor}
\newcommand\ceil[1]{\lceil#1\rceil}

% \newcommand{\comment}[1]{{\sf [#1]}\marginpar[\hfill !!!]{!!!}}
\newcommand{\chris}[1]{\comment{\color{blue}Chris: #1}}
\newcommand{\jan}[1]{\comment{\color{magenta}Jan: #1}}


\begin{document}

\problems

{\sffamily\noindent
We will work on the following exercises together during the work sessions on Tuesday. 

The last questions are marked as homework exercises. 
Your homework must be handed in within one week \textbf{electronically via Canvas before \deadline}. This deadline is strict and late submissions are graded with a 0. At the end of the course, the lowest of all the homework grades will be dropped. 

You are strongly encouraged to work together on the exercises, including the homework. However, after this discussion phase, you have to write down and submit your own individual solution. }

\begin{exercise}[Exercise~3.5 from  $\text{[KL]}$ ]
  Let $|G(s)| = \ell(|s|)$ for some $\ell$. Consider the following experiment:

  \textbf{The PRG indistinguishability experiment} $\mathsf{PRG}_{\mathcal{A},G}(n)$:
  \textit{
  \begin{enumerate}[label=(\alph*)]
\item  A uniform bit $b \in \{0,1\}$ is chosen. If $b = 0$ then choose a uniform $r \in \{0, 1\}^{\ell(n)}$; if $b = 1$ then choose a uniform $s \in \{0,1\}^n$ and set $r := G(s)$.
\item The adversary $\mathcal{A}$ is given $r$, and outputs a bit $b'$.
\item The output of the experiment is defined to be 1 if $b' = b$,
  and 0 otherwise.
  \end{enumerate}
  }
Provide a definition of a pseudorandom generator based on this experiment, and prove that your definition is equivalent to Definition 3.14. (That is, show that $G$ satisfies your definition if and only if it satisfies Definition 3.14.)
  
\end{exercise}

\begin{exercise}[Exercise~3.7 from $\text{[KL]}$ ]
Prove the converse of Theorem 3.18. Namely, show that if $G$ is not a pseudorandom generator then Construction 3.17 does not have indistinguishable encryptions in the presence of an eavesdropper.
\end{exercise}

  


\end{document}
